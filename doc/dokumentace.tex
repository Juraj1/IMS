\documentclass[a4paper,10pt]{article}
%\documentclass[a4paper,10pt]{scrartcl}

\usepackage[utf8]{inputenc}
\usepackage{hyperref}
\usepackage{graphicx}

\hypersetup{
    colorlinks,
    citecolor=black,
    filecolor=black,
    linkcolor=black,
    urlcolor=blue
}

\title{Dokumentace projektu do předmětu \\* Modelování a simulace\\* zadání 8: drůbežárna}
\author{Jiří Zahradník a Veronika Lysáková}
\date{}

\pdfinfo{%
  /Title    ()
  /Author   ()
  /Creator  ()
  /Producer ()
  /Subject  ()
  /Keywords ()
}

\begin{document}

\maketitle
\includegraphics[scale=0.5]{fitnewb.png}
\pagebreak

\renewcommand{\contentsname}{Obsah}
\tableofcontents




\pagebreak

%
% úvod dokumentace
%

\section{Úvod}
V tomto textu bude čtenář seznámen s vytvářením modelu pro diskrétní simulaci 
života broilerů a ekonomické udržitelnosti drůbežárny.
\par 
Je zde popsáno studium autorů, kde autoři získávají potřebné informace
pro vytvoření simulačního modelu na základě reálného modelu drůbežárny.
\par
Na základě experimentů bude demonstrován růst broilerů, jejich úmrtnost
a ekonomická udržitelnost drůbežárny za současných ekonomických podmínek.
A také jak se pohyb ceny kuřat projeví v nákladech podniku. 
\subsection{Zdroje informací pro modelování\cite{modelovani}}
Jako zdroj informací autoři využili oficiální materiály firmy Aviagen\cite{aviagen}.
Jedná se o firmu zajišťující dovoz jednodenních kuřat do drůbežáren ve 130 zemích světa.
Autoři se zaměřili konkrétně na broilery značky Ross\textsuperscript{\textregistered},
jejichž podmínky pro pěstování jsou popsány v dokumentu poskytnutém firmou Aviagen\cite{ross}
na jejich oficiálních webových stránkách. Autoři následně využili tyto informace k
vytvoření abstraktního modelu\cite{abstract_model}, který byl posléze převeden na simulační model\cite{simulation_model}
simulující farmu pro výkrm broilerů Ross\textsuperscript{\textregistered}. 
\subsection{Validace a verifikace}
TODO model zvalidovan pomoci smernic EU


%
% Téma a použité metody
%

\section{Rozbor tématu, použitých metod a technologií}
Drůbežáren může být několik druhů. Patří mezi ně zařízení soustředící 
na produkci nosných slepic, zařízení na produkci vajec, zařízení 
na produkci kuřat určených k rychlému výkrmu, tzv. broilerů a další.
V této modelové studii se autoři zaměřili na zařízení pro produkci broilerů.
\par
Kuřata určena pro výkrm, dovezena do zařízení jsou stará jeden den 
a važí 40 gramů. Prvních 10 dní jsou krmena krmivem typu Starter. Toto krmivo 
musí zajistit dostatečný obsah živin pro kuřata, která v prvních deseti dnech
po vylíhnutí konzumují nejmenší objem krmiva, avšak jejich nároky na výživnost krmiva
jsou největší. Tento fenomén je způsoben přizpůsobování kuřete k životu mimo skořápku.
V tomto stádiu se rozhoduje, zda kuře bude mít dostatečnou chuť k jídlu, zda bude zdravé
a zda bude správně růst.
\par
Druhá část výkrmu je dlouhá 14 dní, od 11. dne po 25. den. V této části se využívá krmivo
Grower, a je podáváno ve formě větších granulí, než tomu bylo u krmiva Starter.
Rapidně se zvyšuje váhový přírustek broilerů, jelikož obsahuje menší procento bílkovin
a aminokyselin, avšak jeho metabolizovatelná nergetická hodnota se zvýšila. Při přechodu
mezi krmivy Starter a Grower je třeba pozorně sledovat kuřata a musí se předejít zmenšení
příjmu potravy nebo oslabení růstu.
\par
Poslední část výkrmu nastává okolo 25. dne věku kuřat. Krmivo Grower nahrazuje krmivo Finisher.
Toto krmivo tvoří největší část příjmu potravy kuřat a tím největší část nákladů na výkrm kuřat.
Proto musí být krmivo Finisher optimalizované na finanční návratnost. Ke konci výkrmu
je třeba vysadit farmaceutické aditiva ať se maso broilerů pročistí 
a nezůstávají v něm rezidua. Avšak je třeba zachovat energetickou hodnotu krmiva z důvodu
udržení hmotnostního přírůstku kuřat. Kuřata jsou po čtyřiceti až dvaačtyřiceti dnech zabita
a jejich maso je zpracováno.


\subsection{Postupy}
TODO\newline
Po nastudování a sepsání potřebných informací autoři vytvořili abstraktní model v podobě
petriho sítě\cite{petriho_sit} pro život kuřete. Viz. obrázek \ref{obr:petri_kure}.\newline


\begin{figure}[h]
\includegraphics[scale=0.4]{kure.png}
\caption{Petriho síť popisující život kuřete}
\label{obr:petri_kure}
\end{figure}



\subsection{Původy metod}
TODO

\section{Koncepce}
TODO

\section{Architektura simulačního modelu}
TODO

\section{Podstata simulačních experimentů a jejich průběh}
TODO

\section{Shrnutí simulačních experimentů a závěr}
TODO


\renewcommand{\figurename}{Obrázek}
\renewcommand{\listoffigures}{Obrázky}
\listoffigures

\renewcommand{\refname}{Zdroje}
\bibliographystyle{unsrt}
\bibliography{dokumentace}
\addcontentsline{toc}{section}{\refname}


\end{document}

\begin{thebibliography}{99}

\bibitem{aviagen}
\bibitem{ross}
\bibitem{concept_model}
\bibitem{simulation_model}
\bibitem{model}
\bibitem{modelovani}
\bibitem{petriho_sit}


\end{thebibliography}