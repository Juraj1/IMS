\documentclass[a4paper,10pt]{article}
%\documentclass[a4paper,10pt]{scrartcl}

\usepackage[utf8]{inputenc}
\usepackage{hyperref}
\usepackage{graphicx}
\usepackage{textcomp}
\usepackage{amsmath}
\usepackage{caption}

\hypersetup{
    colorlinks,
    citecolor=black,
    filecolor=black,
    linkcolor=black,
    urlcolor=blue
}

\title{Dokumentace projektu do předmětu \\* Modelování a simulace\\* zadání 8: drůbežárna}
\author{Jiří Zahradník a Veronika Lysáková}
\date{}

\pdfinfo{%
  /Title    ()
  /Author   ()
  /Creator  ()
  /Producer ()
  /Subject  ()
  /Keywords ()
}

\begin{document}

\maketitle
\includegraphics[scale=0.5]{fitnewb.png}
\pagebreak

\renewcommand{\contentsname}{Obsah}
\tableofcontents


\renewcommand{\figurename}{Obrázek}


\pagebreak

%
% úvod dokumentace
%

\section{Úvod}
V tomto textu bude čtenář seznámen s vytvářením modelu pro diskrétní simulaci\cite{diskretni_simulace}
života broilerů a ekonomické udržitelnosti drůbežárny.
\par 
Je zde popsáno studium autorů, kde autoři získávají potřebné informace
pro vytvoření simulačního modelu na základě reálného modelu drůbežárny.
\par
Na základě experimentů bude demonstrována ekonomická udržitelnost
drůbežárny za současných ekonomických podmínek.

\subsection{Zdroje informací pro modelování\cite{modelovani}}\label{zdroje}
Jako zdroj informací autoři využili oficiální materiály firmy Aviagen\cite{aviagen}.
Jedná se o firmu zajišťující dovoz jednodenních kuřat do drůbežáren ve 130 zemích světa.
Autoři se zaměřili konkrétně na broilery značky Ross\textsuperscript{\textregistered},
jejichž podmínky pro chov jsou popsány v dokumentu poskytnutém firmou Aviagen\cite{ross}
na jejich oficiálních webových stránkách. Autoři následně využili tyto informace k
vytvoření abstraktního modelu\cite{abstract_model}, který byl posléze převeden na simulační model\cite{simulation_model}
simulující farmu pro výkrm broilerů Ross\textsuperscript{\textregistered}. 

\subsection{Validace a verifikace}\label{validaceVerifikace}
Ceny jednotlivých komodit byly ověřeny u oficiálních zdrojů.
Prodejní cena za kg kuřete byla ověřena pomocí Státního intervenčního
zemědělského fondu\cite{szif}, cena vody byla ověřena u vodáren\cite{vodarny}.
Mzdy zaměstnanců byly ověřeny inzeráty na HPP a cena krmiva byla ověřena
u jednotlivých obchodníků. Růst drůbeže a průběh výkrmu byl porovnáván
se zdroji od firmy Aviagen\cite{ross}.


%
% Téma a použité metody
%

\section{Rozbor tématu, použitých metod a technologií}\label{rozbor}
Drůbežáren může být několik druhů. Patří mezi ně drůbežárny soustředící 
na produkci nosných slepic, drůbežárny na produkci vajec, drůbežárny 
na produkci kuřat určených k rychlému výkrmu, tzv. broilerů a další.
V této modelové studii se autoři zaměřili na drůbežárny pro produkci broilerů.
\par
Kuřata určena pro výkrm, dovezena do drůbežárny jsou stará jeden den 
a važí 40 gramů. Prvních 10 dní jsou krmena krmivem typu Starter. Toto krmivo 
musí zajistit dostatečný obsah živin pro kuřata, která v prvních deseti dnech
po vylíhnutí konzumují nejmenší objem krmiva, avšak jejich nároky na výživnost krmiva
jsou největší. Tento fenomén je způsoben přizpůsobování kuřete k životu mimo skořápku.
V tomto stádiu se rozhoduje, zda kuře bude mít dostatečnou chuť k jídlu, zda bude zdravé
a zda bude správně růst.
\par
Druhá část výkrmu je dlouhá 14 dní, od 11. dne po 25. den. V této části se využívá krmivo
Grower, a je podáváno ve formě větších granulí, než tomu bylo u krmiva Starter.
Rapidně se zvyšuje váhový přírustek broilerů, jelikož obsahuje menší procento bílkovin
a aminokyselin, avšak jeho metabolizovatelná energetická hodnota se zvýšila. Při přechodu
mezi krmivy Starter a Grower je třeba pozorně sledovat kuřata a musí se předejít zmenšení
příjmu potravy nebo oslabení růstu.
\par
Poslední část výkrmu nastává okolo 25. dne věku kuřat. Krmivo Grower nahrazuje krmivo Finisher.
Toto krmivo tvoří největší část příjmu potravy kuřat a tím největší část nákladů na výkrm kuřat.
Proto musí být krmivo Finisher optimalizované na finanční návratnost. Ke konci výkrmu
je třeba vysadit farmaceutické aditiva ať se maso broilerů pročistí 
a nezůstávají v něm rezidua. Avšak je třeba zachovat energetickou hodnotu krmiva z důvodu
udržení hmotnostního přírůstku kuřat. Kuřata jsou po čtyřiceti až dvaačtyřiceti dnech zabita
a jejich maso je zpracováno.

\subsection{Postupy}\label{postupy}
Po nastudování a sepsání potřebných informací autoři vytvořili abstraktní model
života kuřete v podobě Petriho sítě\cite{petriho_sit}. Viz. obrázek \ref{obr:petri_kure}.
\par
Pro vytvoření simulačního modelu měli autoři předepsané jazyky C nebo C++. Autoři se rozhodli
pro C++ kvůli možnosti objektového návrhu, který umožňuje provést abstrakci pomocí objektů
a tato abstrakce umožňuje objektu specifické chování na základě vstupních podmínek.
Při porovnání s jazykem C, kde je možno implementovat pseudoobjektový přístup, je objektový 
přístup jazyka C++ intuitivnější a umožňuje jednodušší práci s jednotlivými objekty.
Každé kuře je reprezentováno objektem a s těmito objekty pracuje objekt továrna, který uchovává
data o chovu kuřat. Více o konceptu v kapitole \ref{koncepce}.

\subsection{Původy metod}\label{puvod}
Informace pro práci s jazykem C++ autoři čerpali ze semináře jazyka C++\cite{cpp}
přednášeného doktorem Peringerem. Informace pro práci s objekty byly čerpány
z předmětu IPP\cite{ipp}.





\section{Koncepce}\label{koncepce}
Jelikož nás pouze zajímá ekonomická situace drůbežárny, můžeme z modelu vyloučit detaily
o růstu kuřete. Do modelu také nebyly zahrnuty ceny za energie, jelikož výše poplatků
je vysoce individuální a záleží na každé drůbežárně, jak bude s energiemi hospodařit. Další důvod je 
absence zdrojů, které by určily přesnou spotřebu.
Pro vyhodnocení ekonomické situace nám stačí pouze množství snězeného krmiva,
spotřebované vody, výsledná váha kuřete, počet zpracovaných kuřat, 
kupní cena jednodenního broileru Ross\textsuperscript\textregistered
a velkoobchodní cena vykrmeného broileru.
\par
Koncepční model\cite{concept_model} je vytvořen jako diskrétní model, kde časovou osu reprezentují dny věku kuřete.
Kuře v modelu je reprezentováno jako Markovův proces\cite{markovuv_proces},
kdy každý den proces mění svůj stav na základě svého aktuálního věku.
\par
Život kuřete je modelován Petriho sítí, která zobrazena je na obrázku \ref{obr:petri_kure}.
Po dovezení kuřete začíná dvaačtyřiceti denní výkrm, kde je kuře postupně krmeno, napájeno, v jehož důsledku
přibírá na váze. Kuře taky může umřít před datem zpracování a to je znázorněno přechodem\cite{prechod} s pravděpodobností 0.05\%.
Pokud přežije, přechází do dalšího dne a cyklus se opakuje. Jak bylo již řečeno, časová osa modelu je reprezentována dny
věku kuřat. O posuny vpřed na časové ose se stará drůbežárna, jejichž lze vidět na obrázku \ref{obr:drubezarna}. 
Po zpracování jsou data o spotřebě kuřete předána drůbežárně, ta je vyhodnotí a na tomto základě vytvoří finanční zrávu.

\subsection{Zdůvodnění způsobu vyjádření konceptuálního\newline modelu}
Na obrázku \ref{obr:petri_kure} je zobrazen model života kuřete. Petriho síť byla vybrána, jelikož umožňuje
jednoduché časování přechodů a síť lze bez větších obtíží transformovat na simulační model v jazyce C++, kde přechody
označují volání metod odpovídajícího objektu v simulačním modelu.

\subsection{Konceptuální model}
\begin{figure}[h]
\caption{Petriho síť modelující život kuřete}
\includegraphics[scale=0.4]{kure.png}
\label{obr:petri_kure}
\end{figure}

Na obrázku \ref{obr:drubezarna} můžeme vidět model drůbežárny samotné v podobě Petriho sítě. Můžeme vidět, že po přijeťí kuřat
se model dostane do stavu, kde se rozhoduje, jestli se bude posouvat po časové ose, anebo jestli zpracuje
kuřata. Proměnná A vyjadřuje počet tokenů(teček), které reprezentují množinu kuřat vynásobenou počtem dní výkrmu,
nebo-li: 
$A=\textit{počet kuřat }*\textit{ dny výkrmu}$.
Dokud model iteruje, volá se metoda simulačního modelu \textit{nextDay()}, které posouvá simulační model po časové ose doprava
a až bude proměnná A rovna nule, provede se přechod, který symbolizuje volání metody, která zpracuje kuřata. Následně 
se vytvoří finanční zpráva a simulace se ukončí.


\begin{figure}[h]
\caption{Petriho síť modelující provoz drůbežárny}
\includegraphics[scale=0.4]{drubezarna.png}
\label{obr:drubezarna}
\end{figure}

\pagebreak

\section{Architektura simulačního modelu}\label{architektura}
Architektura simulačního modelu je postavena na objektovém přístupu. Simulační model se skládá ze dvou tříd,
třídy factory, a třídy chicken. Z třídy \textit{factory} se vytvoří objekt drůbežárny, a z třídy \textit{chicken}
se vytvoří jednotlivá kuřata, která se seřadí do vektoru \textit{chick}. S tímto vektorem poté pracuje továrna tak,
že nad ním volá metodu \textit{nextDay()}(lze vidět na obrázku \ref{obr:drubezarna}), která zašle objektu kuřete
zprávu o příchodu následujícího dne. Po přijetí této zprávy objekt kuře simuluje krmení, pití a tloustnutí.
Poté objekt kuře zkontroluje booleovskou hodnotu signalizující, zda je nakaženo a má umřít, anebo žije další den.
Tato booleovská hodnota je nastavena na true v 0.05\% případů. Kontrola se provádí každý den a dohromady to dává ~2\% celkový úhyn.
Pokud kuře uhyne, zvýší se čitač v objektu drůbežárny a s mrtvým objektem kuře se nadále nepracuje.
Po ukončení výkrmu se zavolá metoda drůbežárny \textit{statistics()}, která vyhodnotí údaje o výkrmu a vytvoří finanční analýzu.

\section{Podstata simulačních experimentů a jejich průběh}\label{experimenty}
Podstatou simulačních experimentů je zjišťování, za jakých podmínek je podnik ekonomicky udržitelný.
\subsection{Postup experimentování}
Průběh experimentů probíhal opakovaným spouštěním simulace s různými hodnotami klíčových parametrů, jako
jsou např. ceny krmiva, počet zaměstnanců, počet kuřat, mzdy zaměstnanců a cena krmiva.
Výsledky autoři porovnali a vytvořili z nich závěr o udržitelnosti drůbežárny.

\subsection{Experimenty}
V této kapitole se budeme zabývat vybranými experimenty, na kterých si ukážeme pohyb hranice udržitelnosti.
\subsubsection{Experiment č.1}
Jako vstupní parametry pro tento experiment jsou nastaveny následovně.
Firma má 30 zaměstnanců, jejichž průměrná mzda je dvacet tisíc korun.
Cena krmiva je stanovena pro BR1 na 15 Kč/kg, BR2 na 14 Kč/kg a BR3 na 12 Kč/kg.
Cena vody je stanovena na 0.08104 Kč za litr a do drůbežárny bylo přivezeno 100~000 kuřat při 
ceně 8 Kč za jedno jednodenní kuře.
Po nastavení těchto parametrů byla spuštěna simulace, která měla následující výstupy.
\par
Z 100~000 kuřat se 2457 nedožilo zpracování, tudíž prodat se mohlo pouze 97~543 kuřat o celkové
váze 202~889.44 kilogramů při prodejní ceně 46 korun za kilogram, což dává 9~332~914 Kč.
Přičemž kuřata jsme koupili za 800~000 Kč, na plat zaměstnanců během výkrmu šlo 839~160 Kč.
Cena použitého krmiva BR1 byla 1~254~149 Kč, BR2 byla 1~950~899 Kč a BR3 byla 2 006~639 Kč.
Cena vody byla 68~022 Kč. Z toho nám plyne celkový zisk 2~414~046 Kč.
\subsubsection{Experiment č.2}
Na experimentu číslo 2 chceme ukázat jak se projeví počet nakoupených kuřat na 
zisk drůbežárny. Vstupní parametry jsou nastaveny, až na vyjímku 30~000 nakoupených kuřat, stejně
jako v předchozích experimentech.
\par
Na výsledku tohoto experimentu můžeme vidět, že i při počátečním počtu 30~000 kuřat 
a 29~287 prodaných kuřat může drůbežárna existovat, jelikož zisk je 137~803 Kč.
\subsubsection{Experiment č.3}\label{ex3}
Na experimentu číslo 3 chceme ukázat hranici udržitelnosti drůbežárny bez propouštění
zaměstnanců.
Vstupní parametry jsou nastaveny, až na vyjímku 25~800 nakoupených kuřat, stejně.
\par
Výsledek experimentu nám ukázal, že u hranice udržitelnosti je na 25~800 nakoupených
kuřatech, jelikož počet prodaných kuřat se pohybuje okolo 25~123 kuřat.
\subsubsection{Experiment č.4}
Na experimentu číslo 4 chceme ukázat posunutí hranice udržitelnosti drůbežárny po propuštění
zaměstnanců.
Ve vstupních parametrech provedeme změnu a nastavíme počet zaměstnanců na 20 a posuneme 
počet koupených kuřat na 17 200.
\par
Výsledek experimentu nám ukázal, že u hranice udržitelnosti se sníží na 17~200 nakoupených
kuřat, přičemž jsme prodali 16~792 kuřat. Celkový zisk je 718 Kč, což není mnoho, ale je možno
zaplatit všechny zaměstnance.
\subsubsection{Experiment č.5}
Na exprimentu č. 5 chceme ukázat jaký vliv má na hranici udržitelnosti cena prodáváných kuřat.
Nastavíme počet zaměstnanců opět na 30, jejich průměrný plat stále na 20 000 Kč. Cena krmiva zůstane stejná,
avšak cena za kilogram kuřete vzroste o 10 Kč na 56 Kč a počet kuřat nastavíme na 15~925 kusů.
\par
Můžeme vidět, že při ceně za kilogram kuřete o 10 Kč větší se hranice udržitelnosti snížila na hranici okolo 15~925 kusů,
jelikož výsledná ztráta je 462 Kč.
\subsubsection{Experiment č.6}
Experimentem č. 6 chceme ukázat, jak se projeví zdražení krmiva o pouhé dvě koruny při stejných parametrech jako 
v experimentu 1. BR1 je nastaveno na 17 Kč, BR2 na 16 Kč a BR3 na 14 Kč. Máme opět 100~000 kuřat.
\par
Zisk je oproti experimentu 1 pouze 1~634~860 Kč.
\subsubsection{Experiment č.7}
A nyní se podívejme jak se sníží hranice udržitelnosti. Koupíme 34~000 kuřat.
\par
Výsledky experimentu ukázaly, že zisk je pouze 815 Kč. Lze zde nádherně vidět význam ceny krmiva, a to proto,
že po zdražení krmiva o pouhé 2 koruny musí drůbežárna nakupovat o téměř
10~000 kuřat více, aby bylo možno zaplatit všechny zaměstnance.

\section{Shrnutí simulačních experimentů a závěr}\label{zaver}
V rámci projektu vznikl nástroj, který vychází z obecného modelu pro výkrm
broilerů Ross\textsuperscript\textregistered\cite{ross} a byl implementován v C++.
Tento nástroj by po drobných úpravách byl schopen
simulovat různé situace pro specifické důbežárny a umožnil vedoucím podniků
udržet se v černých číslech.
\par
Simulační experimenty ukázaly jak závisí
pohyb hranice mezi čevenými a černými čísly na jednotlivých parametrech chovu.
Za vskutku zajímavé autoři považují, že cena krmiva se na ziskovosti projeví 
s takovou mírou a o tolik více než počet a mzda zaměstnanců. I když s tímto počítali,
nečekali, že vliv ceny krmiva bude mít takový vliv na zisk a tím i výslednou cenu vykrmeného 
broileru.


\pagebreak
\renewcommand\listfigurename{Seznam obrázků}
\listoffigures
\addcontentsline{toc}{section}{\listfigurename}


\renewcommand{\listtablename}{Seznam tabulek}
\listoftables
\addcontentsline{toc}{section}{\listtablename}


\renewcommand{\refname}{Zdroje}
\bibliographystyle{unsrt}
\bibliography{dokumentace}
\addcontentsline{toc}{section}{\refname}


\end{document}

\begin{thebibliography}{99}

\bibitem{diskretni_simulace}
\bibitem{vodarny}
\bibitem{szif}
\bibitem{markovuv_proces}
\bibitem{aviagen}
\bibitem{ross}
\bibitem{concept_model}
\bibitem{simulation_model}
\bibitem{model}
\bibitem{modelovani}
\bibitem{petriho_sit}
\bibitem{prechod}






\end{thebibliography}